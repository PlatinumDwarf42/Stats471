\documentclass[12pt]{article}
\usepackage{amssymb,amsmath, endnotes,setspace,enumerate,ulem,color,xspace,lscape}
\usepackage[sort]{natbib}
\allowdisplaybreaks[1]
\normalem
\newcommand{\textR}[1]{\textcolor{blue}{\texttt{#1}}}
\newcommand{\R}{\textR{R}}

\begin{document}

\title{Title of your project report}  

\author{
Sean Maloney\thanks{Department of Statistics.  Email: {\tt maloney4@wisc.edu}}  % 1st author
\and 
Sophia Margareta Carayannopoulos\thanks{Department of Statistics.  Email: {\tt  carayannopou@wisc.edu}}  % 2nd author
}

\maketitle

\begin{center}
\textbf{Abstract}
\end{center}

This is the abstract of your report ... 



\thispagestyle{empty}

\newpage

\pagenumbering{arabic} % add page numbers to your report

\section{Introduction}\label{intro}

Cryptocurrencies are a buzzing all over social and mainstream media for many reasons. The technology behind it is interesting but what interest most people is the money that can be made with it. There is a subreddit committed to the discussion and distribution of information about mainstream cryptocurrencies such as Bitcoin and altcoins like Maidsafe. With so much money on the line, we thought it would be interesting to see if anyone has been trying to take advantage through social media. Our goal in this project was to see if we could predict if a cryptocurrency in value would go up using a quantitative measure of media hype.


\section{Statistical Analysis}\label{datagather}

Please add page numbers to your report. Do not split a table across pages. Do not split a word across two lines. It is highly suggested to place the tables and figures close to where they are first mentioned. Please caption figures and tables.


\section{Statistical Analysis}

Please add page numbers to your report. Do not split a table across pages. Do not split a word across two lines. It is highly suggested to place the tables and figures close to where they are first mentioned. Please caption figures and tables.

\section{Conclusions}


Multi-period volatility forecasts feature prominently in asset pricing, portfolio allocation, risk-management and most other areas of finance where long-horizon measures of risk are necessary. Such forecasts can be constructed in three fundamentally different ways. The first approach is to estimate a horizon-specific model of the volatility, such as a weekly or monthly GARCH\, which can then be used to form direct predictions of volatility over the next week, month, etc. The second approach is to estimate a daily model and then iterate forward the daily forecasts to obtain weekly or monthly predictions. The forecasting literature refers to the first approach as ``direct'' and the second as ``iterated''. A third method is the mixed-data sampling (MIDAS)\ approach introduced by \cite{ghysels_etal_midas-jfe}. 


\newpage


%%%%%%%%%%%%%%%%%%%%%%%%%%%%%%%%%%%%%%%%%%%%%%%%%%%%%%%%%%%%%%%%%%%%%%
%%%%%%%%%%%%%% Reference
%%%%%%%%%%%%%%%%%%%%%%%%%%%%%%%%%%%%%%%%%%%%%%%%%%%%%%%%%%%%%%%%%%%%%%
\bibliographystyle{apalike}
\bibliography{referencelist}


\newpage

%%%%%%%%%%%%%%%%%%%%%%%%%%%%%%%%%%%%%%%%%%%%%%%%%%%%%%%%%%%%%%%%%%%%%%
%%%%%%%%%%%%%%  Appendix
%%%%%%%%%%%%%%%%%%%%%%%%%%%%%%%%%%%%%%%%%%%%%%%%%%%%%%%%%%%%%%%%%%%%%%
\appendix
\begin{center}
{\Large {\bf Appendix: \R\ code}}
\end{center}

% INSERT R CODE HERE, 10 point font
{\footnotesize \begin{verbatim}
m <- 10000 
theta.hat <- se <- numeric(5)
g <- function(x) {
  exp(-x - log(1+x^2)) * (x > 0) * (x < 1)
}

x <- runif(m)     #using f0
fg <- g(x)
theta.hat[1] <- mean(fg)
se[1] <- sd(fg)
\end{verbatim} }



\end{document}


